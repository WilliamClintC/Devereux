% Options for packages loaded elsewhere
% Options for packages loaded elsewhere
\PassOptionsToPackage{unicode}{hyperref}
\PassOptionsToPackage{hyphens}{url}
\PassOptionsToPackage{dvipsnames,svgnames,x11names}{xcolor}
%
\documentclass[
  11pt,
]{article}
\usepackage{xcolor}
\usepackage[margin=1in]{geometry}
\usepackage{amsmath,amssymb}
\setcounter{secnumdepth}{5}
\usepackage{iftex}
\ifPDFTeX
  \usepackage[T1]{fontenc}
  \usepackage[utf8]{inputenc}
  \usepackage{textcomp} % provide euro and other symbols
\else % if luatex or xetex
  \usepackage{unicode-math} % this also loads fontspec
  \defaultfontfeatures{Scale=MatchLowercase}
  \defaultfontfeatures[\rmfamily]{Ligatures=TeX,Scale=1}
\fi
\usepackage{lmodern}
\ifPDFTeX\else
  % xetex/luatex font selection
\fi
% Use upquote if available, for straight quotes in verbatim environments
\IfFileExists{upquote.sty}{\usepackage{upquote}}{}
\IfFileExists{microtype.sty}{% use microtype if available
  \usepackage[]{microtype}
  \UseMicrotypeSet[protrusion]{basicmath} % disable protrusion for tt fonts
}{}
\usepackage{setspace}
\makeatletter
\@ifundefined{KOMAClassName}{% if non-KOMA class
  \IfFileExists{parskip.sty}{%
    \usepackage{parskip}
  }{% else
    \setlength{\parindent}{0pt}
    \setlength{\parskip}{6pt plus 2pt minus 1pt}}
}{% if KOMA class
  \KOMAoptions{parskip=half}}
\makeatother
% Make \paragraph and \subparagraph free-standing
\makeatletter
\ifx\paragraph\undefined\else
  \let\oldparagraph\paragraph
  \renewcommand{\paragraph}{
    \@ifstar
      \xxxParagraphStar
      \xxxParagraphNoStar
  }
  \newcommand{\xxxParagraphStar}[1]{\oldparagraph*{#1}\mbox{}}
  \newcommand{\xxxParagraphNoStar}[1]{\oldparagraph{#1}\mbox{}}
\fi
\ifx\subparagraph\undefined\else
  \let\oldsubparagraph\subparagraph
  \renewcommand{\subparagraph}{
    \@ifstar
      \xxxSubParagraphStar
      \xxxSubParagraphNoStar
  }
  \newcommand{\xxxSubParagraphStar}[1]{\oldsubparagraph*{#1}\mbox{}}
  \newcommand{\xxxSubParagraphNoStar}[1]{\oldsubparagraph{#1}\mbox{}}
\fi
\makeatother


\usepackage{longtable,booktabs,array}
\usepackage{calc} % for calculating minipage widths
% Correct order of tables after \paragraph or \subparagraph
\usepackage{etoolbox}
\makeatletter
\patchcmd\longtable{\par}{\if@noskipsec\mbox{}\fi\par}{}{}
\makeatother
% Allow footnotes in longtable head/foot
\IfFileExists{footnotehyper.sty}{\usepackage{footnotehyper}}{\usepackage{footnote}}
\makesavenoteenv{longtable}
\usepackage{graphicx}
\makeatletter
\newsavebox\pandoc@box
\newcommand*\pandocbounded[1]{% scales image to fit in text height/width
  \sbox\pandoc@box{#1}%
  \Gscale@div\@tempa{\textheight}{\dimexpr\ht\pandoc@box+\dp\pandoc@box\relax}%
  \Gscale@div\@tempb{\linewidth}{\wd\pandoc@box}%
  \ifdim\@tempb\p@<\@tempa\p@\let\@tempa\@tempb\fi% select the smaller of both
  \ifdim\@tempa\p@<\p@\scalebox{\@tempa}{\usebox\pandoc@box}%
  \else\usebox{\pandoc@box}%
  \fi%
}
% Set default figure placement to htbp
\def\fps@figure{htbp}
\makeatother





\setlength{\emergencystretch}{3em} % prevent overfull lines

\providecommand{\tightlist}{%
  \setlength{\itemsep}{0pt}\setlength{\parskip}{0pt}}



 
\usepackage[]{biblatex}
\addbibresource{references.bib}


\makeatletter
\@ifpackageloaded{tcolorbox}{}{\usepackage[skins,breakable]{tcolorbox}}
\@ifpackageloaded{fontawesome5}{}{\usepackage{fontawesome5}}
\definecolor{quarto-callout-color}{HTML}{909090}
\definecolor{quarto-callout-note-color}{HTML}{0758E5}
\definecolor{quarto-callout-important-color}{HTML}{CC1914}
\definecolor{quarto-callout-warning-color}{HTML}{EB9113}
\definecolor{quarto-callout-tip-color}{HTML}{00A047}
\definecolor{quarto-callout-caution-color}{HTML}{FC5300}
\definecolor{quarto-callout-color-frame}{HTML}{acacac}
\definecolor{quarto-callout-note-color-frame}{HTML}{4582ec}
\definecolor{quarto-callout-important-color-frame}{HTML}{d9534f}
\definecolor{quarto-callout-warning-color-frame}{HTML}{f0ad4e}
\definecolor{quarto-callout-tip-color-frame}{HTML}{02b875}
\definecolor{quarto-callout-caution-color-frame}{HTML}{fd7e14}
\makeatother
\makeatletter
\@ifpackageloaded{caption}{}{\usepackage{caption}}
\AtBeginDocument{%
\ifdefined\contentsname
  \renewcommand*\contentsname{Table of contents}
\else
  \newcommand\contentsname{Table of contents}
\fi
\ifdefined\listfigurename
  \renewcommand*\listfigurename{List of Figures}
\else
  \newcommand\listfigurename{List of Figures}
\fi
\ifdefined\listtablename
  \renewcommand*\listtablename{List of Tables}
\else
  \newcommand\listtablename{List of Tables}
\fi
\ifdefined\figurename
  \renewcommand*\figurename{Figure}
\else
  \newcommand\figurename{Figure}
\fi
\ifdefined\tablename
  \renewcommand*\tablename{Table}
\else
  \newcommand\tablename{Table}
\fi
}
\@ifpackageloaded{float}{}{\usepackage{float}}
\floatstyle{ruled}
\@ifundefined{c@chapter}{\newfloat{codelisting}{h}{lop}}{\newfloat{codelisting}{h}{lop}[chapter]}
\floatname{codelisting}{Listing}
\newcommand*\listoflistings{\listof{codelisting}{List of Listings}}
\makeatother
\makeatletter
\makeatother
\makeatletter
\@ifpackageloaded{caption}{}{\usepackage{caption}}
\@ifpackageloaded{subcaption}{}{\usepackage{subcaption}}
\makeatother
\usepackage{bookmark}
\IfFileExists{xurl.sty}{\usepackage{xurl}}{} % add URL line breaks if available
\urlstyle{same}
\hypersetup{
  pdftitle={William's Comments},
  pdfauthor={William Clinton Co},
  colorlinks=true,
  linkcolor={blue},
  filecolor={Maroon},
  citecolor={Blue},
  urlcolor={Blue},
  pdfcreator={LaTeX via pandoc}}


\title{William's Comments}
\usepackage{etoolbox}
\makeatletter
\providecommand{\subtitle}[1]{% add subtitle to \maketitle
  \apptocmd{\@title}{\par {\large #1 \par}}{}{}
}
\makeatother
\subtitle{Tariffs, Trade and Tumult Canada's challenges ahead}
\author{William Clinton Co}
\date{July 3, 2025}
\begin{document}
\maketitle
\begin{abstract}
This document provides comments on Prof.~Michael B. Devereux's
presentation, ``Tariffs, Trade, and Tumult: Canada's Challenges Ahead,''
delivered at the Pender Whistler Investment Conference on July 9, 2025.
\end{abstract}

\renewcommand*\contentsname{Table of contents}
{
\hypersetup{linkcolor=}
\setcounter{tocdepth}{3}
\tableofcontents
}
\listoffigures
\listoftables

\setstretch{1.5}
\section{Introduction}\label{sec-introduction}

This document serves as a comprehensive template for creating
professional documents with Quarto. As discussed in
Section~\ref{sec-methods}, we'll explore various features that make
Quarto powerful for academic and professional writing.

\subsection{Document Structure}\label{document-structure}

A well-structured document typically includes:

\begin{enumerate}
\def\labelenumi{\arabic{enumi}.}
\tightlist
\item
  \textbf{Front matter} - Title, author, abstract
\item
  \textbf{Table of contents} - Navigation aid
\item
  \textbf{Main content} - Organized in logical sections
\item
  \textbf{References} - Proper citations and bibliography
\end{enumerate}

\section{Literature Review}\label{sec-literature}

According to recent research \autocite{doe2023}, Quarto represents a
significant advancement in reproducible document generation. The
integration of multiple programming languages makes it particularly
suitable for data science applications \autocite{smith2023}.

\section{Methods}\label{sec-methods}

\subsection{Data Analysis Workflow}\label{data-analysis-workflow}

Our analysis follows these steps:

\begin{enumerate}
\def\labelenumi{\arabic{enumi}.}
\tightlist
\item
  Data collection and cleaning
\item
  Exploratory data analysis\\
\item
  Statistical modeling
\item
  Results visualization
\end{enumerate}

\subsubsection{Sample Code}\label{sample-code}

Here's an example of data analysis code that would be executed:

\subsection{Statistical Analysis}\label{statistical-analysis}

The relationship can be expressed mathematically as:

\begin{equation}\phantomsection\label{eq-regression}{
Y = \beta_0 + \beta_1 X + \epsilon
}\end{equation}

where \(\epsilon \sim N(0, \sigma^2)\) represents the error term.

\section{Results}\label{sec-results}

\subsection{Summary Statistics}\label{summary-statistics}

Table~\ref{tbl-summary} presents the descriptive statistics for our
variables.

\begin{longtable}[]{@{}lllll@{}}
\caption{Summary Statistics}\label{tbl-summary}\tabularnewline
\toprule\noalign{}
Variable & Mean & SD & Min & Max \\
\midrule\noalign{}
\endfirsthead
\toprule\noalign{}
Variable & Mean & SD & Min & Max \\
\midrule\noalign{}
\endhead
\bottomrule\noalign{}
\endlastfoot
X & 0.12 & 0.98 & -2.4 & 2.1 \\
Y & 0.24 & 2.1 & -4.8 & 4.2 \\
\end{longtable}

\subsection{Model Results}\label{model-results}

The regression analysis (see Equation~\ref{eq-regression}) yielded the
following results:

\begin{itemize}
\tightlist
\item
  Intercept: \(\beta_0 = 0.05\) (SE = 0.21, p = 0.81)
\item
  Slope: \(\beta_1 = 1.98\) (SE = 0.19, p \textless{} 0.001)
\item
  R² = 0.78
\end{itemize}

As shown in \textbf{?@fig-scatter}, there's a strong positive
relationship between the variables.

\section{Discussion}\label{sec-discussion}

\subsection{Key Findings}\label{key-findings}

Our analysis reveals several important insights:

\begin{tcolorbox}[enhanced jigsaw, bottomtitle=1mm, colframe=quarto-callout-note-color-frame, opacityback=0, leftrule=.75mm, opacitybacktitle=0.6, bottomrule=.15mm, colback=white, toptitle=1mm, colbacktitle=quarto-callout-note-color!10!white, breakable, coltitle=black, left=2mm, title=\textcolor{quarto-callout-note-color}{\faInfo}\hspace{0.5em}{Important Note}, arc=.35mm, toprule=.15mm, rightrule=.15mm, titlerule=0mm]

The strong correlation observed in our data suggests a meaningful
relationship that warrants further investigation.

\end{tcolorbox}

\begin{tcolorbox}[enhanced jigsaw, bottomtitle=1mm, colframe=quarto-callout-warning-color-frame, opacityback=0, leftrule=.75mm, opacitybacktitle=0.6, bottomrule=.15mm, colback=white, toptitle=1mm, colbacktitle=quarto-callout-warning-color!10!white, breakable, coltitle=black, left=2mm, title=\textcolor{quarto-callout-warning-color}{\faExclamationTriangle}\hspace{0.5em}{Limitation}, arc=.35mm, toprule=.15mm, rightrule=.15mm, titlerule=0mm]

This analysis is based on simulated data and should not be interpreted
as real research findings.

\end{tcolorbox}

\subsection{Future Work}\label{future-work}

Potential extensions of this work include:

\begin{itemize}
\tightlist
\item[$\square$]
  Collect real-world data
\item[$\square$]
  Implement more sophisticated models
\item[$\square$]
  Conduct sensitivity analyses
\item[$\square$]
  Validate findings with independent datasets
\end{itemize}

\section{Conclusion}\label{sec-conclusion}

This document demonstrates the power of Quarto for creating
professional, reproducible documents. The combination of narrative text,
executable code, mathematical equations, and proper citations makes it
an excellent choice for academic and professional writing.

Key advantages of using Quarto include:

\begin{itemize}
\tightlist
\item
  \textbf{Reproducibility}: Code and results are integrated
\item
  \textbf{Flexibility}: Multiple output formats supported
\item
  \textbf{Professional appearance}: High-quality typesetting
\item
  \textbf{Cross-references}: Automatic numbering and linking
\end{itemize}


\printbibliography[title=References]



\end{document}

% Options for packages loaded elsewhere
% Options for packages loaded elsewhere
\PassOptionsToPackage{unicode}{hyperref}
\PassOptionsToPackage{hyphens}{url}
\PassOptionsToPackage{dvipsnames,svgnames,x11names}{xcolor}
%
\documentclass[
  11pt,
]{article}
\usepackage{xcolor}
\usepackage[margin=1in]{geometry}
\usepackage{amsmath,amssymb}
\setcounter{secnumdepth}{5}
\usepackage{iftex}
\ifPDFTeX
  \usepackage[T1]{fontenc}
  \usepackage[utf8]{inputenc}
  \usepackage{textcomp} % provide euro and other symbols
\else % if luatex or xetex
  \usepackage{unicode-math} % this also loads fontspec
  \defaultfontfeatures{Scale=MatchLowercase}
  \defaultfontfeatures[\rmfamily]{Ligatures=TeX,Scale=1}
\fi
\usepackage{lmodern}
\ifPDFTeX\else
  % xetex/luatex font selection
\fi
% Use upquote if available, for straight quotes in verbatim environments
\IfFileExists{upquote.sty}{\usepackage{upquote}}{}
\IfFileExists{microtype.sty}{% use microtype if available
  \usepackage[]{microtype}
  \UseMicrotypeSet[protrusion]{basicmath} % disable protrusion for tt fonts
}{}
\usepackage{setspace}
\makeatletter
\@ifundefined{KOMAClassName}{% if non-KOMA class
  \IfFileExists{parskip.sty}{%
    \usepackage{parskip}
  }{% else
    \setlength{\parindent}{0pt}
    \setlength{\parskip}{6pt plus 2pt minus 1pt}}
}{% if KOMA class
  \KOMAoptions{parskip=half}}
\makeatother
% Make \paragraph and \subparagraph free-standing
\makeatletter
\ifx\paragraph\undefined\else
  \let\oldparagraph\paragraph
  \renewcommand{\paragraph}{
    \@ifstar
      \xxxParagraphStar
      \xxxParagraphNoStar
  }
  \newcommand{\xxxParagraphStar}[1]{\oldparagraph*{#1}\mbox{}}
  \newcommand{\xxxParagraphNoStar}[1]{\oldparagraph{#1}\mbox{}}
\fi
\ifx\subparagraph\undefined\else
  \let\oldsubparagraph\subparagraph
  \renewcommand{\subparagraph}{
    \@ifstar
      \xxxSubParagraphStar
      \xxxSubParagraphNoStar
  }
  \newcommand{\xxxSubParagraphStar}[1]{\oldsubparagraph*{#1}\mbox{}}
  \newcommand{\xxxSubParagraphNoStar}[1]{\oldsubparagraph{#1}\mbox{}}
\fi
\makeatother


\usepackage{longtable,booktabs,array}
\usepackage{calc} % for calculating minipage widths
% Correct order of tables after \paragraph or \subparagraph
\usepackage{etoolbox}
\makeatletter
\patchcmd\longtable{\par}{\if@noskipsec\mbox{}\fi\par}{}{}
\makeatother
% Allow footnotes in longtable head/foot
\IfFileExists{footnotehyper.sty}{\usepackage{footnotehyper}}{\usepackage{footnote}}
\makesavenoteenv{longtable}
\usepackage{graphicx}
\makeatletter
\newsavebox\pandoc@box
\newcommand*\pandocbounded[1]{% scales image to fit in text height/width
  \sbox\pandoc@box{#1}%
  \Gscale@div\@tempa{\textheight}{\dimexpr\ht\pandoc@box+\dp\pandoc@box\relax}%
  \Gscale@div\@tempb{\linewidth}{\wd\pandoc@box}%
  \ifdim\@tempb\p@<\@tempa\p@\let\@tempa\@tempb\fi% select the smaller of both
  \ifdim\@tempa\p@<\p@\scalebox{\@tempa}{\usebox\pandoc@box}%
  \else\usebox{\pandoc@box}%
  \fi%
}
% Set default figure placement to htbp
\def\fps@figure{htbp}
\makeatother





\setlength{\emergencystretch}{3em} % prevent overfull lines

\providecommand{\tightlist}{%
  \setlength{\itemsep}{0pt}\setlength{\parskip}{0pt}}



 
\usepackage[]{biblatex}
\addbibresource{references.bib}


\usepackage{hyperref}
\hypersetup{
  colorlinks=true,
  linkcolor=blue,
  urlcolor=blue,
  breaklinks=true,
  pdfborder={0 0 0}
}
\makeatletter
\@ifpackageloaded{caption}{}{\usepackage{caption}}
\AtBeginDocument{%
\ifdefined\contentsname
  \renewcommand*\contentsname{Table of contents}
\else
  \newcommand\contentsname{Table of contents}
\fi
\ifdefined\listfigurename
  \renewcommand*\listfigurename{List of Figures}
\else
  \newcommand\listfigurename{List of Figures}
\fi
\ifdefined\listtablename
  \renewcommand*\listtablename{List of Tables}
\else
  \newcommand\listtablename{List of Tables}
\fi
\ifdefined\figurename
  \renewcommand*\figurename{Figure}
\else
  \newcommand\figurename{Figure}
\fi
\ifdefined\tablename
  \renewcommand*\tablename{Table}
\else
  \newcommand\tablename{Table}
\fi
}
\@ifpackageloaded{float}{}{\usepackage{float}}
\floatstyle{ruled}
\@ifundefined{c@chapter}{\newfloat{codelisting}{h}{lop}}{\newfloat{codelisting}{h}{lop}[chapter]}
\floatname{codelisting}{Listing}
\newcommand*\listoflistings{\listof{codelisting}{List of Listings}}
\makeatother
\makeatletter
\makeatother
\makeatletter
\@ifpackageloaded{caption}{}{\usepackage{caption}}
\@ifpackageloaded{subcaption}{}{\usepackage{subcaption}}
\makeatother
\usepackage{bookmark}
\IfFileExists{xurl.sty}{\usepackage{xurl}}{} % add URL line breaks if available
\urlstyle{same}
\hypersetup{
  pdftitle={William's Update},
  pdfauthor={William Clinton Co},
  colorlinks=true,
  linkcolor={blue},
  filecolor={Maroon},
  citecolor={blue},
  urlcolor={blue},
  pdfcreator={LaTeX via pandoc}}


\title{William's Update}
\usepackage{etoolbox}
\makeatletter
\providecommand{\subtitle}[1]{% add subtitle to \maketitle
  \apptocmd{\@title}{\par {\large #1 \par}}{}{}
}
\makeatother
\subtitle{Identifying Related Papers}
\author{William Clinton Co}
\date{July 24, 2025}
\begin{document}
\maketitle
\begin{abstract}
This document compiles and ranks related literature by relevance to
``Productivity and Wedges: Economic Convergence and the Real Exchange
Rate'' (Devereux, Fujiwara, \& Granados, 2025). More litearture will
follow.
\end{abstract}

\renewcommand*\contentsname{Table of contents}
{
\hypersetup{linkcolor=}
\setcounter{tocdepth}{3}
\tableofcontents
}

\setstretch{1}
\section{Introduction}\label{introduction}

This is an updated document, representing William's current work on
identifying related literature.

We have also sent a Zotero library invitation to Michael.Devereux@ubc.ca
(invited on 2025-07-10 16:56:43), which provides access to the current
literature we are reviewing.

Any feedback is appreciated.

\subsection{Recent Papers}\label{recent-papers}

\textbf{Links:}

\begin{itemize}
\tightlist
\item
  \href{https://chahrour.github.io/papers/Disconnect_revisited.pdf?utm_source=chatgpt.com}{Disconnect
  Revisited -- Chahrour et al.}
\item
  \href{https://link.springer.com/article/10.1057/s41308-024-00251-0}{Springer
  Article}
\item
  \href{https://www.aeaweb.org/articles?from=f&id=10.1257\%2Fmac.20210445&utm_source=chatgpt.com}{AEA
  Journal Article}
\item
  \href{https://bpb-us-e2.wpmucdn.com/sites.utdallas.edu/dist/8/1090/files/2024/08/Tech-Dollars-and-Exchange-Rate-Reconnect.pdf?utm_source=chatgpt.com}{Tech
  Dollars and Exchange Rate Reconnect -- UTD}
\end{itemize}

\begin{center}\rule{0.5\linewidth}{0.5pt}\end{center}

\subsubsection{Exchange Rate Disconnect
Revisited}\label{exchange-rate-disconnect-revisited}

\textbf{Authors:} Ryan Chahrour (Cornell University), Vito Cormun (Santa
Clara University), Pierre De Leo (University of Maryland), Pablo
Guerrón-Quintana (Boston College), Rosen Valchev (Boston College)\\
\textbf{Date:} November 10, 2023\\
\textbf{Links:}
\href{https://drive.google.com/file/d/1Di78Mg3vjuy0XLX3APJZYbzAxVlVmKp-/view?usp=sharing}{Working
Paper}\\
\textbf{Notes:}

\begin{itemize}
\tightlist
\item
\end{itemize}

\section{Related Literature (Ranked by
Relevance)}\label{related-literature-ranked-by-relevance}

\subsection{High Relevance}\label{high-relevance}

\subsubsection{Understanding Korea's Long-Run Real Exchange Rate
Behavior}\label{understanding-koreas-long-run-real-exchange-rate-behavior}

\textbf{Authors:} Douglas A. Irwin, Maurice Obstfeld\\
\textbf{Date:} August 2024\\
\textbf{Links:}
\href{https://www.nber.org/system/files/working_papers/w32769/w32769.pdf}{NBER
Working Paper}\\
\textbf{Notes:}

\begin{itemize}
\tightlist
\item
  Challenges the Harrod-Balassa-Samuelson hypothesis,
\item
  Decomposes real exchange rate changes into internal and external
  components.
\item
  Suggests that Korea's policy of intervening to smooth nominal exchange
  rate fluctuations may be justified
\end{itemize}

\begin{center}\rule{0.5\linewidth}{0.5pt}\end{center}

\subsubsection{The Predictive Power of Equilibrium Exchange Rate
Models}\label{the-predictive-power-of-equilibrium-exchange-rate-models}

\textbf{Authors:} Michele Ca' Zorzi (European Central Bank), Adam Cap
(Bank for International Settlements), Andrej Mijakovic (European
University Institute), Michał Rubaszek (Warsaw School of Economics)\\
\textbf{Date:} January 2020\\
\textbf{Links:}
\href{https://drive.google.com/file/d/1XJWcFNb5DNIUlFdC4Kl4DVPB8pWIEm3y/view?usp=sharing}{Google
Drive},
\href{https://papers.ssrn.com/sol3/papers.cfm?abstract_id=3515882}{SSRN}\\
\textbf{Notes:}

\begin{itemize}
\tightlist
\item
  Evaluates the predictive performance of three equilibrium exchange
  rate models---PPP, BEER, and MB.
\item
  Finds that PPP provides the most reliable forecasts.
\item
  BEER fails to outperform PPP.
\item
  The MB approach, while theoretically rich, performs the worst in
  forecasting.
\item
  Robustness checks confirm that model adjustments do not significantly
  improve prediction accuracy.
\end{itemize}

\begin{center}\rule{0.5\linewidth}{0.5pt}\end{center}

\subsubsection{Measuring the Balassa-Samuelson Effect: A Guidance Note
on the RPROD
Database}\label{measuring-the-balassa-samuelson-effect-a-guidance-note-on-the-rprod-database}

\textbf{Authors:} Cécile Couharde, Anne-Laure Delatte, Carl Grekou,
Valérie Mignon, Florian Morvillier\\
\textbf{Date:} May 2020\\
\textbf{Links:}
\href{https://drive.google.com/file/d/16L1DDmRDnJ33SYlDcjePIBteyV5TkbHU/view?usp=sharing}{Google
Drive},
\href{https://www.sciencedirect.com/science/article/pii/S2110701720300417}{ScienceDirect}\\
\textbf{Notes:}

\begin{itemize}
\tightlist
\item
  Introduces the RPROD database developed by CEPII to complement the
  EQCHANGE dataset.
\item
  Serves as an empirical tool for analyzing the Balassa-Samuelson
  effect.
\end{itemize}

\begin{center}\rule{0.5\linewidth}{0.5pt}\end{center}

\subsubsection{The Determinants of Real Exchange Rates in Transition
Economies}\label{the-determinants-of-real-exchange-rates-in-transition-economies}

\textbf{Authors:} Dan Meshulam, Peter Sanfey (EBRD)\\
\textbf{Date:} 2019\\
\textbf{Links:}
\href{https://drive.google.com/file/d/1MldCJB2yPD3AEI93kt9qSX11rjcTdySo/view?usp=sharing}{Google
Drive}\\
\textbf{Notes:}

\begin{itemize}
\tightlist
\item
  Empirical analysis of real exchange rate drivers in transition
  economies.
\item
  Strong support for the Balassa-Samuelson hypothesis.
\item
  RER aligns with productivity trends more than other variables like
  capital inflows or government consumption.
\end{itemize}

\begin{center}\rule{0.5\linewidth}{0.5pt}\end{center}

\subsubsection{In Search of Dominant Drivers of the Real Exchange
Rate}\label{in-search-of-dominant-drivers-of-the-real-exchange-rate}

\textbf{Authors:} Wataru Miyamoto, Thuy Lan Nguyen, Hyunseung Oh\\
\textbf{Date:} May 8, 2025\\
\textbf{Links:}
\href{https://drive.google.com/file/d/1v1P90qYEBl_1tm_MXqShOefqqOm5Fr2j/view?usp=sharing}{Google
Drive}\\
\textbf{Notes:}

\begin{itemize}
\tightlist
\item
  Examines the dominant macroeconomic drivers of real exchange rates and
  macro aggregates at business cycle frequencies across G7 countries.
\item
  Dominant drivers of the real exchange rate are orthogonal to the main
  drivers of business cycles.
\end{itemize}

\begin{center}\rule{0.5\linewidth}{0.5pt}\end{center}

\subsubsection{The Story of the Real Exchange
Rate}\label{the-story-of-the-real-exchange-rate}

\textbf{Authors:} Oleg Itskhoki\\
\textbf{Date:} 2021\\
\textbf{Links:}
\href{https://drive.google.com/file/d/1vhVbum9po2xoiaPT8suUqOqjDg5o8EkI/view?usp=sharing}{Google
Drive}\\
\textbf{Notes:}

\begin{itemize}
\tightlist
\item
  \emph{Must Read.} Presents a comprehensive general equilibrium
  framework for real exchange rate (RER) determination.
\item
  Mechanisms such as home bias, incomplete pass-through, expenditure
  switching, goods market clearing, imperfect international risk
  sharing, fiscal constraints, and monetary policy regimes.
\item
  Discusses issues related to the stationarity and predictability of
  exchange rates.
\end{itemize}

\begin{center}\rule{0.5\linewidth}{0.5pt}\end{center}

\subsection{Medium to High Relevance}\label{medium-to-high-relevance}

\subsubsection{Real Exchange Rate Misalignments and Currency Crises in
the Former Soviet Union
Countries}\label{real-exchange-rate-misalignments-and-currency-crises-in-the-former-soviet-union-countries}

\textbf{Authors:} Viktar Dudzich\\
\textbf{Date:} November 25, 2021\\
\textbf{Links:}
\href{https://drive.google.com/file/d/1O9A9N2GmYd0LRPhnE3LdUbVbDA0hd6zk/view?usp=sharing}{Google
Drive}\\
\textbf{Notes:}

\begin{itemize}
\tightlist
\item
  Estimates equilibrium exchange rates for 10 former Soviet republics
  using BEER and NATREX concepts with a pooled mean group estimator.
\item
  Compares estimated misalignments before, during, and after currency
  crisis episodes, and examines their relationship to crisis-related
  variables.
\item
  Journal is not as highly ranked.
\end{itemize}

\begin{center}\rule{0.5\linewidth}{0.5pt}\end{center}

\subsubsection{What is Needed for Convergence? The Role of Capital and
Finance}\label{what-is-needed-for-convergence-the-role-of-capital-and-finance}

\textbf{Authors:} Bryan Hardy, Can Sever\\
\textbf{Date:} June 13, 2025\\
\textbf{Links:}
\href{https://drive.google.com/file/d/1UEEN58-yuPErOCtVsMKjYnqeX62k0BU4/view?usp=sharing}{Google
Drive}\\
\textbf{Notes:}

\begin{center}\rule{0.5\linewidth}{0.5pt}\end{center}

\subsubsection{Housing Rent, Inelastic Housing Supply, and International
Business
Cycles}\label{housing-rent-inelastic-housing-supply-and-international-business-cycles}

\textbf{Author:} Seungyub Han\\
\textbf{Date:} September 21, 2024\\
\textbf{Links:}
\href{https://drive.google.com/file/d/1loPvpeqirt1pulEqajxQNMCGLeIFt_AS/view?usp=sharing}{Google
Drive}\\
\textbf{Notes:}

\begin{itemize}
\tightlist
\item
  Connects to the Balassa-Samuelson effect to housing
\end{itemize}

\begin{center}\rule{0.5\linewidth}{0.5pt}\end{center}

\subsection{Medium Relevance}\label{medium-relevance}

\subsubsection{Piecing the Puzzle: Real Exchange Rates and Long-Run
Fundamentals}\label{piecing-the-puzzle-real-exchange-rates-and-long-run-fundamentals}

\textbf{Authors:} Hilde C. Bjørnland, Leif Brubakk, Nicolò
Maffei-Faccioli\\
\textbf{Date:} February 2025\\
\textbf{Links:}
\href{https://drive.google.com/file/d/1r4zwdG1oeC5fSfW7nBtjWwDWBQFNNTmZ/view?usp=sharing}{Google
Drive}\\
\textbf{Notes:}

\begin{itemize}
\tightlist
\item
  Examines the structural determinants of real exchange rates, focusing
  on persistent low-frequency movements.
\end{itemize}

\begin{center}\rule{0.5\linewidth}{0.5pt}\end{center}

\subsubsection{Temperature and Real Exchange
Rates}\label{temperature-and-real-exchange-rates}

\textbf{Authors:} Yue Gu, Jing Zhang, Xiaohui Liu\\
\textbf{Date:} 11 July 2025\\
\textbf{Links:}
\href{https://drive.google.com/file/d/1mEMh-d8eDPpk9WaKuob5Wz34xlObrjSt/view?usp=sharing}{Google
Drive}\\
\textbf{Notes:}

\begin{itemize}
\tightlist
\item
  Incorporates average temperature into the Balassa--Samuelson model.
\item
  Examines the impact of temperature differences on real exchange rates
  (RERs).
\end{itemize}

\begin{center}\rule{0.5\linewidth}{0.5pt}\end{center}

\subsubsection{The Price of Development: The Penn--Balassa--Samuelson
Effect
Revisited}\label{the-price-of-development-the-pennbalassasamuelson-effect-revisited}

\textbf{Authors:} Fadi Hassan\\
\textbf{Date:} 2016\\
\textbf{Links:}
\href{https://drive.google.com/file/d/1bWMV2iq0ZnGvyt1L25UzIOewSEYH5uBC/view?usp=sharing}{Google
Drive}\\
\textbf{Notes:}

\begin{itemize}
\tightlist
\item
  Revisits the Penn effect, showing a non-linear price--income
  relationship that turns negative for low-income countries.
\item
  Suggests structural transformation stages as key to explaining RER
  variation in developing economies.
\item
  Robust to PPP measurement bias.
\item
  May be less directly relevant, as its primary focus is on developing
  countries.
\end{itemize}

\begin{center}\rule{0.5\linewidth}{0.5pt}\end{center}

\subsubsection{Competitiveness and Productivity in the Baltics: Common
Shocks, Different
Implications}\label{competitiveness-and-productivity-in-the-baltics-common-shocks-different-implications}

\textbf{Authors:} Saioa Armendariz, Carlos de Resende, Alice Fan,
Gianluigi Ferrucci, Bingjie Hu, Sadhna Naik, Can Ugur\\
\textbf{Date:} January 17, 2025\\
\textbf{Links:}
\href{https://drive.google.com/file/d/1GFUm5ckExKBUJ7CXxZ4ZyplTKT2Krc28/view?usp=sharing}{Google
Drive}\\
\textbf{Notes:}

\begin{itemize}
\tightlist
\item
  Although focused on competitiveness, it uses multivariate filtering
  consistent with Balassa-Samuelson.
\item
  Relevant Quote: ``Multivariate filtering techniques and estimates of
  the real effective exchange rates based on historical productivity
  trends, consistent with Balassa-Samuelson, confirm that differences in
  long-term total factor productivity growth have affected external
  competitiveness.''
\end{itemize}

\begin{center}\rule{0.5\linewidth}{0.5pt}\end{center}

\subsubsection{The Real Exchange Rate in the Long Run: Balassa-Samuelson
Effects
Reconsidered}\label{the-real-exchange-rate-in-the-long-run-balassa-samuelson-effects-reconsidered}

\textbf{Authors:} Michael D. Bordo, Ehsan U. Choudhri, Giorgio Fazio,
Ronald MacDonald\\
\textbf{Date:} 2017\\
\textbf{Links:}
\href{https://doi.org/10.1016/j.jimonfin.2017.03.011}{DOI},
\href{https://drive.google.com/file/d/1wpLTPo2OgVXe54w8JiDwyBwYTGqkBz4I/view?usp=sharing}{Google
Drive}\\
\textbf{Notes:}

\begin{itemize}
\tightlist
\item
  Uses historical data to investigate the Balassa-Samuelson effect over
  the long run.\\
\item
  Demonstrates significant variation in productivity effects across
  different monetary regimes.
\end{itemize}

\begin{center}\rule{0.5\linewidth}{0.5pt}\end{center}

\subsubsection{Real Effective Exchange Rate Misalignment in the Euro
Area: A Counterfactual
Analysis}\label{real-effective-exchange-rate-misalignment-in-the-euro-area-a-counterfactual-analysis}

\textbf{Authors:} Makram El-Shagi, Axel Lindner, Gregor von Schweinitz\\
\textbf{Date:} November 26, 2015\\
\textbf{Links:}
\href{https://drive.google.com/file/d/1rg_R7y6ti0Y7HCTl6L9Alxj7nUcQf6Cn/view?usp=sharing}{Google
Drive}\\
\textbf{Notes:}

\begin{itemize}
\tightlist
\item
  (REER) misalignments within the Euro area using synthetic matching
  methods.
\item
  Constructs counterfactuals for member states to quantify the extent of
  misalignment.
\item
  Crisis countries align with a mix of advanced and emerging economy
  profiles.
\item
  Highly relevant to the literature on Eurozone imbalances.
\item
  Mid-tier journal.
\end{itemize}

\begin{center}\rule{0.5\linewidth}{0.5pt}\end{center}

\subsubsection{The `Real' Explanation of the PPP
Puzzle}\label{the-real-explanation-of-the-ppp-puzzle}

\textbf{Authors:} Nicholas Ford, Charles Yuji Horioka\\
\textbf{Date:} April 2016\\
\textbf{Links:} \href{http://www.nber.org/papers/w22198}{NBER},
\href{https://drive.google.com/file/d/1dIKOGSL5vl_e6A86TPpEyXVHKGyGa3yG/view?usp=sharing}{Google
Drive}\\
\textbf{Notes:}

\begin{itemize}
\tightlist
\item
  Argues that global financial markets alone are insufficient to enable
  net capital transfers and real interest rate equalization across
  countries.
\end{itemize}

\begin{center}\rule{0.5\linewidth}{0.5pt}\end{center}

\subsection{Low to Medium Relevance}\label{low-to-medium-relevance}

\subsubsection{Exchange Rate Models Are Better Than You Think, and Why
They Didn't Work in the Old
Days}\label{exchange-rate-models-are-better-than-you-think-and-why-they-didnt-work-in-the-old-days}

\textbf{Authors:} Charles Engel, Steve Pak Yeung Wu\\
\textbf{Date:} 2024\\
\textbf{Links:}
\href{https://drive.google.com/file/d/1bKyyCi1qhqifzHM9Sx3ahU_tNOCwdYmz/view?usp=sharing}{Google
Drive},
\href{https://www.nber.org/system/files/working_papers/w32808/w32808.pdf}{NBER}\\
\textbf{Notes:}

\begin{itemize}
\tightlist
\item
  Explores why traditional exchange rate models now better fit U.S.
  data.
\item
  Argues that inflation-targeting and improved monetary policy reduce
  self-fulfilling expectations, improving model fit post-2000s.
\end{itemize}

\begin{center}\rule{0.5\linewidth}{0.5pt}\end{center}

\subsubsection{The Purchasing Power Parity and Exchange-Rate Economics
Half a Century
On}\label{the-purchasing-power-parity-and-exchange-rate-economics-half-a-century-on}

\textbf{Authors:} Hai Long Vo, Duc Hong Vo\\
\textbf{Date:} 2023\\
\textbf{Links:}
\href{https://drive.google.com/file/d/1GvSoUtKJiy58EFfif0aoFeuvEYS26tqs/view?usp=sharing}{Google
Drive}\\
\textbf{Notes:}

\begin{itemize}
\tightlist
\item
  A survey paper providing historical and theoretical context for
  exchange-rate economics.\\
\item
  Not directly relevant but may be useful for broader insight.
\end{itemize}

\begin{center}\rule{0.5\linewidth}{0.5pt}\end{center}

\subsubsection{The Balassa-Samuelson effect reversed: new evidence from
OECD
countries}\label{the-balassa-samuelson-effect-reversed-new-evidence-from-oecd-countries}

\textbf{Authors:} Matthias Gubler \& Christoph Sax\\
\textbf{Date:} 27 February 2019\\
\textbf{Links:}
\href{https://drive.google.com/file/d/1_555bkCnPqh45YBhcRcNZMX4dL5exJOW/view?usp=sharing}{Full
Text}\\
\textbf{Notes:}

\begin{itemize}
\tightlist
\item
  Reconsiders the Balassa-Samuelson (BS) hypothesis using an OECD
  country panel from 1970 to 2008.
\item
  Compares three data sets on sectoral productivity, including newly
  constructed data on total factor productivity.
\item
  Within- and between-dimension estimation results do not support the BS
  hypothesis.
\item
  Mid-tier journal.
\end{itemize}

\begin{center}\rule{0.5\linewidth}{0.5pt}\end{center}

\subsection{Low Relevance}\label{low-relevance}

\subsubsection{Subnational Purchasing Power of Parity in OECD Countries:
Estimates Based on the Balassa-Samuelson
Hypothesis}\label{subnational-purchasing-power-of-parity-in-oecd-countries-estimates-based-on-the-balassa-samuelson-hypothesis}

\textbf{Authors:} Alex Costa, Jaume Garcia, Josep Lluís Raymond, Daniel
Sánchez-Serra\\
\textbf{Date:} 2019\\
\textbf{Links:}
\href{https://drive.google.com/file/d/1QT0iR2TwIY5ayraL7pAe0Ir_R03U17nR/view?usp=sharing}{Google
Drive}\\
\textbf{Notes:}

\begin{itemize}
\tightlist
\item
  Examines regional PPP estimation within OECD countries using a method
  based on the Balassa-Samuelson hypothesis.
\item
  Methodologically adjacent, but limited applicability to national RER
  studies.
\end{itemize}

\begin{center}\rule{0.5\linewidth}{0.5pt}\end{center}

\subsubsection{Real Exchange Rate Misalignments in the Euro
Area}\label{real-exchange-rate-misalignments-in-the-euro-area}

\textbf{Authors:} Michael Fidora, Claire Giordano, Martin Schmitz\\
\textbf{Date:} July 28, 2020\\
\textbf{Links:}
\href{https://drive.google.com/file/d/1r1e4j018-tVvMPwVqttye_ftEfZTTNsj/view?usp=sharing}{Google
Drive}\\
\textbf{Notes:}

\begin{itemize}
\tightlist
\item
  Tackles the same topic of real exchange rate misalignments from a Euro
  Area perspective.
\item
  While the journal is not highly ranked, the paper offers relevant
  comparative insights for understanding regional dynamics in real
  exchange rate adjustments.
\end{itemize}

\begin{center}\rule{0.5\linewidth}{0.5pt}\end{center}

\subsubsection{Is the Balassa-Samuelson Hypothesis Still Relevant?
Cross-Country Evidence from
1950--2017}\label{is-the-balassa-samuelson-hypothesis-still-relevant-cross-country-evidence-from-19502017}

\textbf{Authors:} Mohammed Ershad Hussain, Mahfuzul Haque\\
\textbf{Date:} September 28, 2020\\
\textbf{Links:}
\href{https://drive.google.com/file/d/1UOk4LsRAddTO--0tr4ulynOaXQht7Agv/view?usp=sharing}{Google
Drive}, \href{https://doi.org/10.1453/jepe.v7i3.2096}{DOI}\\
\textbf{Notes:}

\begin{itemize}
\tightlist
\item
  182 countries. Employs Arellano-Bond dynamic panel data estimation.
\item
  Low ranked journal.
\end{itemize}

\begin{center}\rule{0.5\linewidth}{0.5pt}\end{center}

\subsubsection{Real Exchange Rate Fundamentals: A Synthesis of the
Literature}\label{real-exchange-rate-fundamentals-a-synthesis-of-the-literature}

\textbf{Authors:} Oluremi Davies Ogun (University of Ibadan)\\
\textbf{Date:} 2019\\
\textbf{Links:}
\href{https://drive.google.com/file/d/1SIhMMtsybhiY3eJg-oGcCAvoqJPDKXtQ/view?usp=sharing}{Google
Drive}, \href{https://doi.org/10.6000/1929-7092.2019.08.36}{DOI}\\
\textbf{Notes:}

\begin{itemize}
\tightlist
\item
  Synthesis of approaches to modeling equilibrium real exchange rates.
\item
  Limited reputation.
\end{itemize}

\begin{center}\rule{0.5\linewidth}{0.5pt}\end{center}

\subsubsection{Is the Balassa-Samuelson Hypothesis Still Relevant?
Cross-Country Evidence from
1950--2017}\label{is-the-balassa-samuelson-hypothesis-still-relevant-cross-country-evidence-from-19502017-1}

\textbf{Authors:} Mohammed Ershad Hussain, Mahfuzul Haque\\
\textbf{Date:} September 28, 2020\\
\textbf{Links:}
\href{https://drive.google.com/file/d/1UOk4LsRAddTO--0tr4ulynOaXQht7Agv/view?usp=sharing}{Google
Drive},\href{https://doi.org/10.1453/jepe.v7i3.2096}{DOI}\\
\textbf{Notes:}

\begin{itemize}
\tightlist
\item
  182 countries. Employs Arellano-Bond dynamic panel data estimation.
\item
  Low ranked journal.
\end{itemize}

\begin{center}\rule{0.5\linewidth}{0.5pt}\end{center}

\subsubsection{Real Exchange Rate Fundamentals: A Synthesis of the
Literature}\label{real-exchange-rate-fundamentals-a-synthesis-of-the-literature-1}

\textbf{Authors:} Oluremi Davies Ogun (University of Ibadan)\\
\textbf{Date:} 2019\\
\textbf{Links:}
\href{https://drive.google.com/file/d/1SIhMMtsybhiY3eJg-oGcCAvoqJPDKXtQ/view?usp=sharing}{Google
Drive}, \href{https://doi.org/10.6000/1929-7092.2019.08.36}{DOI}\\
\textbf{Notes:}

\begin{itemize}
\tightlist
\item
  Synthesis of approaches to modeling equilibrium real exchange rates.
\item
  Limited reputation.
\end{itemize}


\printbibliography



\end{document}
